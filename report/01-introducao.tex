\section{Indrodução}

Conforme a evolução dos meios de comunicação, a capacidade de armazenamento, processamento e transmissão de dados aumentaram exponencialmente. 
de apenas um servidor tornou-se insuficiente para atender a demanda de usuários das grandes plataformas 
de serviços digitais de maneira efetiva e de alta disponibilidade.

Nesse contexto, para o projeto da disciplina SCC0243 - Arquitetura de Sistemas Gerenciadores de Base de Dados,
decidimos comparar a performance de um banco de dados distribuído com um banco de dados centralizado,
utilizando o PostgreSQL e o Citus, uma extensão do PostgreSQL para bancos de dados distribuídos.

O objetivo do projeto é implementar e avaliar as diferenças de performance e overheads de uma base de dados
relacional distribuída do Citus <em diversas configurações>, em comparação ao PostgreSQL. 
Para isso, utilizaremos benchmarks padronizados pelo Concelho de Desempenho de Processamento de Transações(TPC, em inglês) 
para diferentes cenários de carga de trabalho.