\section{Fundamentação Teórica}
\subsection{Banco de Dados Distribuído}
Um banco de dados distribuído é um banco de dados cujos dados estão armazenados em diferentes locais, 
mas se intercomunicam de forma a parecerem um único banco de dados para o usuário.

A distribuição dos dados entre várias máquinas possibilita melhorar a escalabilidade,
a disponibilidade e a tolerância a falhas do sistema. Entretanto, essa distribuição 
implica em custos adicionais de comunicação e sincronização entre os nós do sistema.

\subsection{Fragmentação e Replicação de Dados}
Acerca da distribuição dos dados em bancos de dados distribuídos, 
tanto as tuplas como os atributos que as compõem podem ser fragmentados entre os nós do sistema,
e cópias desses dados podem ser replicadas em mais de um nó. 

\subsubsection{Fragmentação Horizontal}
A fragmentação horizontal consiste das distribuição das <tuplas inteiras> de uma tabela entre as diferentes máquinas do sistema.
Essa fragmentação é feita para que cada fragmento contenha um subconjunto das tuplas da tabela original,
e cada fragmento contém todas as colunas da tabela original.

\subsubsection{Fragmentação Vertical}
A fragmentação vertical consiste na distribuição dos atributos de uma tabela entre as diferentes máquinas do sistema, 
mantendo em comum a chave primária da tabela original.
Essa fragmentação é feita para que cada fragmento armazene um subconjunto dos atributos para cada tupla da tabela original.

\subsubsection{Fragmentação Mista}
A fragmentação mista combina as estratégias de fragmentação horizontal e vertical.
Nesse caso, cada fragmento armazena um subconjuntos dos atributos de uma tabela para algumas das tuplas.

\subsubsection{Replicação}
No que diz respeito a redundância dos dados entre os nós do sistema,
a replicação consiste na cópia de um fragmento de dados em mais de um nó. 
Essa cópia pode ser feita de forma parcial ou completa, e uma maior redundância dos dados 
implca em maior disponibilidade e tolerância a falhas do sistema, mas aumenta os custos de armazenamento e sincronização.

<Escrever sobre sincronização talvez>

\subsection{Homogeneidade e Heterogeneidade}
<>

\subsection{Consulta em Banco de Dados Distribuído}
\subsubsection{Plano de Consulta}


\subsection{Transações}
\subsubsection{Tolerância a Falhas}

