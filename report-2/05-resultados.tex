\section{Experimentos e Resultados}

Para comparar o desempenho das arquiteturas propostas, realizamos o benchmark YCSB com as configurações do Cassandra descritas na Seção de Sistema Proposto.
Os resultados estão descritos a seguir, com foco em throughput, latência e overheads de cada configuração.


\subsection{Resultados do Benchmark YCSB}
	\begin{table}[H]
    \centering
    \caption{Resultados do benchmark YCSB - Único nó}
    \begin{tabular}{lcccccc}
    \hline
    Operação & Operações & Latência Média (\textmu s) & Latência Mín (\textmu s) & Latência Máx (\textmu s) \\
    \hline
    INSERT   & 10\,000\,000 & 1\,222,82 & 197   & 148\,607 \\
    READ     & 5\,000\,753  & 2\,296,55 & 220   & 135\,295 \\
    UPDATE   & 4\,999\,247  & 1\,681,11 & 159   & 149\,759 \\
    \hline
    \end{tabular}
    \end{table}

    \begin{table}[H]
    \centering
    \caption{Resultados agregados do benchmark YCSB - 3 nós sem replicação}
    \begin{tabular}{lrrrrrr}
    \hline
    Operação & Operações & Latência Média (\textmu s) & Latência Mín (\textmu s) & Latência Máx (\textmu s) \\
    \hline
    INSERT   & 10\,000\,000 & 703,01 & 159 & 139\,135 \\
    READ     & 4\,998\,961  & 759,82 & 210 & 97\,151  \\
    UPDATE   & 5\,001\,039  & 656,09 & 160 & 95\,935  \\
    \hline
    \end{tabular}
    \end{table}

    \begin{table}[H]
    \centering
    \caption{Resultados agregados do benchmark YCSB - 3 nós com replicação} 
    \begin{tabular}{lrrrrrr}
    \hline
    Operação & Operações & Latência Média (\textmu s) & Latência Mín (\textmu s) & Latência Máx (\textmu s) \\
    \hline
    INSERT   & 10\,000\,000 & 1\,514,36 & 186 & 149\,119  \\
    READ     & 5\,000\,119  & 2\,334,83 & 268 & 125\,119  \\
    UPDATE   & 4\,999\,881  & 1\,170,27 & 182 & 114\,431  \\
    \hline
    \end{tabular}
    \end{table}

\subsection{Análise Comparativa dos Resultados}

A análise dos resultados obtidos evidencia diferenças significativas no desempenho das configurações avaliadas.
O cluster com três nós sem replicação apresentou as menores latências médias para todas as operações (INSERT, READ e UPDATE),
demonstrando maior eficiência na distribuição da carga e no processamento das requisições.

A configuração com replicação replicação no cluster de três nós resultou em um aumento considerável nas médias das latências,
especialmente nas operações de escrita e leitura. 
Esse aumento pode ser atribuído ao overhead inerente ao mecanismo de replicação, que demanda sincronização adicional entre os nós do cluster.

Comparando-se o desempenho do nó único com as demais configurações,
observa-se que o nó único apresentou as maiores latências médias em todas as operações, 
com exceção da operação UPDATE em relção ao cluster com replicação, cuja latência média (1,170,27~$\mu$s) foi inferior à do nó único (1,681,11~$\mu$s).

De modo geral, obsevamos que o overhead causado pela replicação é significativo,
mas é um custo necessário em aplicações que exijam maior disponibilidade e tolerância a falhas.
Para ambientes em que a latência é o principal critério de desempenho, a configuração de cluster sem replicação se mostra mais adequada.

\subsubsection{Análise dos Recursos Utilizados no Cluster}
Além das métricas de latência e throughput, decidimos comparar o uso de recursos do cluster durante a execução do benchmark YCSB.
Abaixo estão os gráficos que mostram o uso de CPU, memória e rede durante a execução do benchmark, para a configuração sem e com replicação dos dados.

\begin{figure}[H]
	\captionsetup{labelformat=empty}
  	\caption{Sem Replicação}
	\centering
  \begin{minipage}{0.32\linewidth}
    \centering
    \includegraphics[width=\linewidth]{imgs/3-1-01.png}
    \caption{Nó 1}
  \end{minipage}
  \hfill
  \begin{minipage}{0.32\linewidth}
    \centering
    \includegraphics[width=\linewidth]{imgs/3-1-02.png}
    \caption{Nó 2}
  \end{minipage}
  \hfill
  \begin{minipage}{0.32\linewidth}
    \centering
    \includegraphics[width=\linewidth]{imgs/3-1-03.png}
    \caption{Nó 3}
  \end{minipage}
\end{figure}

\begin{figure}[H]
	\captionsetup{labelformat=empty}
	\caption{Com Replicação}
  \centering
  \begin{minipage}{0.32\linewidth}
    \centering
    \includegraphics[width=\linewidth]{imgs/3-3-01.png}
    \caption{Nó 1}
  \end{minipage}
  \hfill
  \begin{minipage}{0.32\linewidth}
    \centering
    \includegraphics[width=\linewidth]{imgs/3-3-02.png}
    \caption{Nó 2}
  \end{minipage}
  \hfill
  \begin{minipage}{0.32\linewidth}
    \centering
    \includegraphics[width=\linewidth]{imgs/3-3-03.png}
    \caption{Nó 3}
  \end{minipage}
\end{figure}

Comparando o gráfico do uso de recursos do cluster sem replicação com o do cluster com replicação, observamos que para as arquiteturas utilizadas,
com excessão do uso de memória semelhante, todas métricas apresentam um aumento significativo no uso de recursos quando a replicação é ativada.
As métricas de leitura e escrita de disco e de uso de rede são as que mais se destacam, sendo que os clusters com replicação superaram em mais do dobro do uso desses recursos pelos cluster sem replicação.