\section{Experimentos e Resultados}

Para comparar o desempenho das arquiteturas propostas, realizamos o benchmark YCSB com as configurações do Cassandra descritas na Seção \ref{sec:arquitetura}.
Os resultados estão descritos a seguir, com foco em throughput, latência e overheads de cada configuração.


\subsection{Resultados do Benchmark YCSB}

	\begin{table}[H]
    \centering
    \caption{Resultados do benchmark YCSB - Único nó}
    \begin{tabular}{lcccccc}
    \hline
    Operação & Operações & Latência Média (\textmu s) & Latência Mín (\textmu s) & Latência Máx (\textmu s) \\
    \hline
    INSERT   & 10\,000\,000 & 1\,222,82 & 197   & 148\,607 \\
    READ     & 5\,000\,753  & 2\,296,55 & 220   & 135\,295 \\
    UPDATE   & 4\,999\,247  & 1\,681,11 & 159   & 149\,759 \\
    \hline
    \end{tabular}
    \end{table}

    \begin{table}[H]
    \centering
    \caption{Resultados agregados do benchmark YCSB - 3 nós sem replicação}
    \begin{tabular}{lrrrrrr}
    \hline
    Operação & Operações & Latência Média (\textmu s) & Latência Mín (\textmu s) & Latência Máx (\textmu s) \\
    \hline
    INSERT   & 10\,000\,000 & 703,01 & 159 & 139\,135 \\
    READ     & 4\,998\,961  & 759,82 & 210 & 97\,151  \\
    UPDATE   & 5\,001\,039  & 656,09 & 160 & 95\,935  \\
    \hline
    \end{tabular}
    \end{table}

    \begin{table}[H]
    \centering
    \caption{Resultados agregados do benchmark YCSB - 3 nós com replicação} 
    \begin{tabular}{lrrrrrr}
    \hline
    Operação & Operações & Latência Média (\textmu s) & Latência Mín (\textmu s) & Latência Máx (\textmu s) \\
    \hline
    INSERT   & 10\,000\,000 & 1\,514,36 & 186 & 149\,119  \\
    READ     & 5\,000\,119  & 2\,334,83 & 268 & 125\,119  \\
    UPDATE   & 4\,999\,881  & 1\,170,27 & 182 & 114\,431  \\
    \hline
    \end{tabular}
    \end{table}

\subsection{Análise Comparativa dos Resultados}

A análise dos resultados obtidos evidencia diferenças significativas no desempenho das configurações avaliadas.
O cluster com três nós sem replicação apresentou as menores latências médias para todas as operações (INSERT, READ e UPDATE),
demonstrando maior eficiência na distribuição da carga e no processamento das requisições.

A configuração com replicação replicação no cluster de três nós resultou em um aumento considerável nas médias das latências,
especialmente nas operações de escrita e leitura. 
Esse aumento pode ser atribuído ao overhead inerente ao mecanismo de replicação, que demanda sincronização adicional entre os nós do cluster.

Comparando-se o desempenho do nó único com as demais configurações,
observa-se que o nó único apresentou as maiores latências médias em todas as operações, 
com exceção da operação UPDATE em relção ao cluster com replicação, cuja latência média (1,170,27~$\mu$s) foi inferior à do nó único (1,681,11~$\mu$s).

De modo geral, obsevamos o overhead causado pela replicação é significativo,
mas é um custo necessário em aplicações que exijam maior disponibilidade e tolerância a falhas.
Para ambientes em que a latência é o principal critério de desempenho, a configuração de cluster sem replicação se mostra mais adequada.